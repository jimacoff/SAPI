\def \skipIntro{Sauter l'introduction}
\def \indexFooter{Répertoire des espèces CITES}
\def \historyFooter{Annales des inscriptions à la CITES}
\def \indexIntroductionFile{../../public/latex/index_introduction_fr.pdf}
\def \historyIntroductionFile{../../public/latex/history_introduction_fr.pdf}

\def \annotationsKey{Clé des annotations}
\def \nonHashAnnotations{Annontations non précédées de "\#"}
\def \hashAnnotations{Annotations précédées de "\#"}
\def \historicalSummaryOfAnnotations{HISTORIQUE DES ANNOTATIONS}
\def \hashAnnotationsIndexInfo{
Les annotations sont utilisées dans les annexes CITES pour indiquer les populations et les parties et produits concernés par l'inscription ou pour préciser la portée de celle-ci. La signification des annotations précédées du signe dièse (\#), qui concerne uniquement la flore, a changé au fil des ans. Les annotations précédées de \# actuellement en vigueur ont été adoptées à la 16\superscript{e} session de la Conférence des Parties (CoP16) et sont expliquées ci-dessous.
}
\def \hashAnnotationsHistoryInfo{
Les annotations sont utilisées dans les annexes CITES pour indiquer les populations et les parties et produits concernés par l'inscription ou pour préciser la portée de celle-ci. La signification des annotations a changé au fil des ans. Nous en présentons ci-dessous un historique, avec la date à laquelle elles sont devenues applicables et la session de la Conférence des Parties (CoP) à laquelle elles ont été adoptées. Ainsi, l'annotation "\#1" désignait à l'origine simplement les "racines", alors qu'elle signifie aujourd'hui:
\vspace{15pt}

\begin{adjustwidth}{1.5cm}{}
Toutes les parties et tous les produits sauf:
\begin{enumerate}[a)]
\item les graines, les spores et le pollen (y compris les pollinies);
\item les semis et les cultures de tissus obtenus in vitro, en milieu solide ou liquide, transportés dans des conteneurs stériles;
\item les fleurs coupées provenant de plantes reproduites artificiellement; et
\item les fruits, et leurs parties et produits, provenant de plantes reproduites artificiellement du genre \textit{Vanilla}.
\end{enumerate}
\end{adjustwidth}
\vspace{15pt}

Les dates de validité indiquées dans le tableau ci-dessous se réfèrent aux dates d'entrée en vigueur des nouvelles Annexes I et II, c'est-à-dire 90 jours après chaque session de la Conférence des Parties. En revanche, des amendements à l'Annexe III pouvant être faits à n'importe quel moment, il se peut que les annotations liées à des espèces inscrites à cette Annexe soient entrées en vigueur plus de 90 jours après la session de la Conférence des Parties sous laquelle elles apparaissent. Ainsi, l'annotation \#13 indiquée sous "CoP15" (valables à compter du 23 juin 2010) est liée à l'inscription d'une espèce à l'Annexe III qui n'est entrée en vigueur que le 14 octobre 2010. De tels cas sont rares, la date d'entrée en vigueur de toute annotation est indiquée dans ces annales à côté de l'espèce concernée.
}
\def \validFrom{Valables à compter du}
