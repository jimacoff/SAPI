\def \skipIntro{Saltar introducción}
\def \indexFooter{Índice de especies CITES}
\def \historyFooter{Historia de las inclusiones en los Apéndices de la CITES}
\def \indexIntroductionFile{../../public/latex/index_introduction_es.pdf}
\def \historyIntroductionFile{../../public/latex/history_introduction_es.pdf}

\def \annotationsKey{Clave de anotaciones}
\def \nonHashAnnotations{Anotaciones no precedidas por "\#"}
\def \hashAnnotations{Anotaciones precedidas por "\#"}
\def \historicalSummaryOfAnnotations{RESUMEN HISTÓRICO DE LAS ANOTACIONES}
\def \hashAnnotationsIndexInfo{
ES Annotations are used in the CITES Appendices to indicate which population, parts or derivatives are concerned by the listing or to clarify its scope. The meaning of the \# annotations (applicable to flora only) has changed over the years. The \# annotations that are currently valid are those adopted at the 16th Conference of the Parties (CoP 16). These are provided below.
}
\def \hashAnnotationsHistoryInfo{
Las anotaciones en los Apéndices CITES se utilizan para indicar la población, las partes o los derivados concernidos por la inclusión, o para aclarar su alcance. El significado de las anotaciones ha cambiado a lo largo de los años. A continuación se presenta un resumen histórico de esas anotaciones, con la fecha en que entraron en vigor y la reunión de la Conferencia de las Partes (CoP) en que fueron adoptadas. Por ejemplo, la anotación "\#1" designaba originalmente solo las raíces, mientras que ahora designa:
\vspace{15pt}

\begin{adjustwidth}{1.5cm}{}
Todas las partes y derivados, excepto:
\begin{enumerate}[a)]
\item las semillas, las esporas y el polen (inclusive las polinias);
\item los cultivos de plántulas o de tejidos obtenidos in vitro, en medios sólidos o líquidos, que se transportan en envases estériles;
\item las flores cortadas de plantas reproducidas artificialmente; y
\item los frutos, y sus partes y derivados, de plantas reproducidas artificialmente del género \textit{Vanilla}.
\end{enumerate}
\end{adjustwidth}
\vspace{15pt}

Las fechas de validez indicadas en el cuadro infra se refieren a las fechas de entrada en vigor de los nuevos Apéndices I y II, a saber, 90 días después de cada reunión de la Conferencia de las Partes (CoP). Dado que las inclusiones en el Apéndice III pueden hacerse en todo momento, las anotaciones específicas a las especies incluidas en ese Apéndice pueden haber entrado en vigor ulteriormente a los 90 días después de la CoP bajo la que aparecen. Por ejemplo, la anotación \#13 incluida bajo la sección correspondiente a la CoP15 infra (en vigor a partir del 23 de junio de 2010) está vinculada a la inclusión de una especie en el Apéndice III, que entró en vigor el 14 de octubre de 2010. Estos casos son más bien raros y, en cada caso, la fecha de entrada en vigor de la anotación se indica en la Historia de las anotaciones a los Apéndices de la CITES, al lado de la especie a la que se aplica.
}
\def \validFrom{En vigor a partir del}
