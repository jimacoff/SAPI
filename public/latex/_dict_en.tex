\def \skipIntro{Skip Introduction}
\def \indexFooter{Index of CITES Species}
\def \historyFooter{History of CITES listings}
\def \indexIntroductionFile{../../public/latex/index_introduction_en.pdf}
\def \historyIntroductionFile{../../public/latex/history_introduction_en.pdf}
\def \hashAnnotationsHistoryInfo{
Annotations are used in the CITES Appendices to indicate which population, parts or derivatives are concerned
by the listing or to clarify its scope. The meaning of these annotations has changed over the years. Below is a historical summary of these meanings, with the date when they became valid and the meeting of the Conference of the Parties (CoP) at which they were adopted. For instance, annotation "\#1" originally simply designated "roots", while it now designates:


All parts and derivatives, except:

a) seeds, spores and pollen (including pollinia);

b) seedling or tissue cultures obtained in vitro, in solid or liquid media, transported in sterile containers;

c) cut flowers of artificially propagated plants; and

d) fruits, and parts and derivatives thereof, of artificially propagated plants of the genus Vanilla.

The validity dates indicated in the table below refer to the dates of entry into effect of new Appendices I and II, that is 90 days after each meeting of the Conference of Parties (CoP). Since inclusions in Appendix
III may be made at any time, however, annotations specific to species included in that Appendix may have come into force later than 90 days after the CoP they appear under. For instance, annotation \#13 listed under the CoP15 section below (valid from 23 June 2010) is linked to the inclusion of a species in Appendix III that entered into force on 14 October 2010. Such cases are very rare and, in any case, the date of entry into force of any annotation is indicated the History of CITES Listings next to the species it applies to.
}
